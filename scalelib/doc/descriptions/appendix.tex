\chapter{The detal numerics}
\section{4th order central differnce}
{
\footnotesize
The 4th order central difference is given by
\begin{eqnarray}
&& \frac{\partial \phi}{\partial x}
=\frac{-\phi_{i+2}+8\phi_{i+1}-8\phi_{i-1}+\phi_{i+2}}{12\Delta x} = 0
\end{eqnarray}
where
\begin{eqnarray}
&& \phi_{i+2} = \phi_{i}
+ 2 \Delta x \left(\frac{\partial \phi}{\partial x}\right)_{i}
+ 2 \Delta x^2 \left(\frac{\partial^2 \phi}{\partial x^2}\right)_{i}
+ \frac{4 \Delta x^3}{3} \left(\frac{\partial^3 \phi}{\partial x^3}\right)_{i}
+ \frac{2 \Delta x^4}{3} \left(\frac{\partial^3 \phi}{\partial x^4}\right)_{i}
+ O(\Delta x^5)\\
&& \phi_{i+1} = \phi_{i}
+ \Delta x \left(\frac{\partial \phi}{\partial x}\right)_{i}
+ \frac{\Delta x^2}{2} \left(\frac{\partial^2 \phi}{\partial x^2}\right)_{i}
+ \frac{\Delta x^3}{6} \left(\frac{\partial^3 \phi}{\partial x^3}\right)_{i}
+ \frac{\Delta x^4}{24} \left(\frac{\partial^3 \phi}{\partial x^4}\right)_{i}
+ O(\Delta x^5)\\
&& \phi_i = \phi_i\\
&& \phi_{i-1} = \phi_{i}
- \Delta x \left(\frac{\partial \phi}{\partial x}\right)_{i}
+ \frac{\Delta x^2}{2} \left(\frac{\partial^2 \phi}{\partial x^2}\right)_{i}
- \frac{\Delta x^3}{6} \left(\frac{\partial^3 \phi}{\partial x^3}\right)_{i}
+ \frac{\Delta x^4}{24} \left(\frac{\partial^3 \phi}{\partial x^4}\right)_{i}
+ O(\Delta x^5)\\
&& \phi_{i+2} = \phi_{i}
- 2 \Delta x \left(\frac{\partial \phi}{\partial x}\right)_{i}
+ 2 \Delta x^2 \left(\frac{\partial^2 \phi}{\partial x^2}\right)_{i}
- \frac{4 \Delta x^3}{3} \left(\frac{\partial^3 \phi}{\partial x^3}\right)_{i}
+ \frac{2 \Delta x^4}{3} \left(\frac{\partial^3 \phi}{\partial x^4}\right)_{i}
+ O(\Delta x^5)
\end{eqnarray}
Therefore,
\begin{eqnarray}
&&
\frac{-\phi_{i+2}+8\phi_{i+1}-8\phi_{i-1}+\phi_{i+2}}{12\Delta x}
= \left(\frac{\partial \phi}{\partial x}\right)_{i}+ O(\Delta x^4)\\
&&
\frac{(-\phi_{i+2}+7\phi_{i+1}+7\phi_{i}-\phi_{i-1})-
(-\phi_{i+1}+7\phi_{i}+7\phi_{i-1}-\phi_{i-2})}{12\Delta x}
= \left(\frac{\partial \phi}{\partial x}\right)_{i}+ O(\Delta x^4)
\end{eqnarray}


\section{Flux Corrected Transport scheme}
Equation (\ref{eq:tracer_int}) can be written as
\begin{eqnarray}
%%
\left(\rho q\right)^{n+1}_{i,j,k} 
&=& \left(\rho q\right)^{n}_{i,j,k}
- \frac{1}{\Delta x \Delta y \Delta z}
\big[ \nonumber\\
&+&
\left[ C_{i+\frac{1}{2},j,k} F_{i+\frac{1}{2},j,k}^{high}
+ \left( 1 - C_{i+\frac{1}{2},j,k}\right) F_{i+\frac{1}{2},j,k}^{low}\right]\nonumber\\
&-&
\left[ C_{i-\frac{1}{2},j,k} F_{i-\frac{1}{2},j,k}^{high}
+ \left( 1 - C_{i-\frac{1}{2},j,k}\right) F_{i-\frac{1}{2},j,k}^{low}\right]\nonumber\\
&+&
\left[ C_{i,j+\frac{1}{2},k} F_{i,j+\frac{1}{2},k}^{high}
+ \left( 1 - C_{i,j+\frac{1}{2},k}\right) F_{i,j+\frac{1}{2},k}^{low}\right]\nonumber\\
&-&
\left[ C_{i,j-\frac{1}{2},k} F_{i,j-\frac{1}{2},k}^{high}
+ \left( 1 - C_{i,j-\frac{1}{2},k}\right) F_{i,j-\frac{1}{2},k}^{low}\right]\nonumber\\
&+&
\left[ C_{i,j,k+\frac{1}{2}} F_{i,j,k+\frac{1}{2}}^{high}
+ \left( 1 - C_{i,j,k+\frac{1}{2}}\right) F_{i,j,k+\frac{1}{2}}^{low}\right]\nonumber\\
&-&
\left[ C_{i,j,k-\frac{1}{2}} F_{i,j,k-\frac{1}{2}}^{high}
+ \left( 1 - C_{i,j,k-\frac{1}{2}}\right) F_{i,j,k-\frac{1}{2}}^{low}\right]\nonumber\\
\big]
\label{eq:1step_integ_tracer}
\end{eqnarray}
where
\begin{eqnarray}
  F_{i+\frac{1}{2},j,k}^{high,low} &=& \Delta t \Delta y \Delta z (\rho u)_{i+\frac{1}{2},j,k} q_{i+\frac{1}{2},j,k}^{high,low}\\
  F_{i,j+\frac{1}{2},k}^{high,low} &=& \Delta t \Delta z \Delta x (\rho u)_{i,j+\frac{1}{2},k} q_{i,j+\frac{1}{2},k}^{high,low}\\
  F_{i,j,k+\frac{1}{2}}^{high,low} &=& \Delta t \Delta x \Delta y (\rho u)_{i,j,k+\frac{1}{2}} q_{i,j,k+\frac{1}{2}}^{high,low}
\end{eqnarray}
The anti-diffusive flux are defined as
\begin{eqnarray}
  A_{i+\frac{1}{2},j,k} &=& F_{i+\frac{1}{2},j,k}^{high}-F_{i+\frac{1}{2},j,k}^{low}\\
  A_{i,j+\frac{1}{2},k} &=& F_{i,j+\frac{1}{2},k}^{high}-F_{i,j+\frac{1}{2},k}^{low}\\
  A_{i,j,k+\frac{1}{2}} &=& F_{i,j,k+\frac{1}{2}}^{high}-F_{i,j,k+\frac{1}{2}}^{low}
\end{eqnarray}

Equation (\ref{eq:1step_integ_tracer}) can be rewritten as
\begin{eqnarray}
%%
\left(\rho q\right)^{n+1}_{i,j,k} 
&=& \left(\rho q\right)^{n}_{i,j,k}
- \frac{1}{\Delta x \Delta y \Delta z}
\big[ \nonumber\\
&+&
\left[ F_{i+\frac{1}{2},j,k}^{low}
+ C_{i+\frac{1}{2},j,k}  A_{i+\frac{1}{2},j,k}\right]\nonumber\\
&-&
\left[ F_{i-\frac{1}{2},j,k}^{low}
+ C_{i-\frac{1}{2},j,k}  A_{i-\frac{1}{2},j,k}\right]\nonumber\\
&+&
\left[ F_{i,j+\frac{1}{2},k}^{low}
+ C_{i,j+\frac{1}{2},k}  A_{i,j+\frac{1}{2},k}\right]\nonumber\\
&-&
\left[ F_{i,j-\frac{1}{2},k}^{low}
+ C_{i,j-\frac{1}{2},k}  A_{i,j-\frac{1}{2},k}\right]\nonumber\\
&+&
\left[ F_{i,j,k+\frac{1}{2}}^{low}
+ C_{i,j,k+\frac{1}{2}}  A_{i,j,k+\frac{1}{2}}\right]\nonumber\\
&-&
\left[ F_{i,j,k-\frac{1}{2}}^{low}
+ C_{i,j,k-\frac{1}{2}}  A_{i,j,k-\frac{1}{2}}\right]\nonumber\\
\big]
\label{eq:1step_integ_tracer2}
\end{eqnarray}
In practice, we calculate Eq.(\ref{eq:1step_integ_tracer2}) by the 
following steps:
\begin{enumerate}
%
\item The tentative values are calculated by using the low order flux:
\begin{eqnarray}
\left(\rho q\right)^{\dagger}_{i,j,k} 
&=& \left(\rho q\right)^{n}_{i,j,k}\nonumber\\
&-& \frac{1}{\Delta x \Delta y \Delta z}
\left[
+ F_{i+\frac{1}{2},j,k}^{low}-F_{i-\frac{1}{2},j,k}^{low}
+ F_{i,j+\frac{1}{2},k}^{low}-F_{i,j-\frac{1}{2},k}^{low}
+ F_{i,j,k+\frac{1}{2}}^{low}-F_{i,j,k-\frac{1}{2}}^{low}
\right]
\end{eqnarray}
%
\item Allowable maximum and minimum values are calculated:
\begin{eqnarray}
\left(\rho q\right)^{\max}_{i,j,k}
&=& \max [\nonumber\\
&&\max( \left(\rho q\right)^{\dagger}_{i,j,k},\left(\rho q\right)^{n}_{i,j,k} ),\nonumber\\
&&\max( \left(\rho q\right)^{\dagger}_{i-1,j,k},\left(\rho q\right)^{n}_{i-1,j,k} ),\nonumber\\
&&\max( \left(\rho q\right)^{\dagger}_{i+1,j,k},\left(\rho q\right)^{n}_{i+1,j,k} ),\nonumber\\
&&\max( \left(\rho q\right)^{\dagger}_{i,j-1,k},\left(\rho q\right)^{n}_{i,j-1,k} ),\nonumber\\
&&\max( \left(\rho q\right)^{\dagger}_{i,j+1,k},\left(\rho q\right)^{n}_{i,j+1,k} ),\nonumber\\
&&\max( \left(\rho q\right)^{\dagger}_{i,j,k-1},\left(\rho q\right)^{n}_{i,j,k-1} ),\nonumber\\
&&\max( \left(\rho q\right)^{\dagger}_{i,j,k+1},\left(\rho q\right)^{n}_{i,j,k+1} ) \nonumber\\
&&]\\
\left(\rho q\right)^{\min}_{i,j,k}
&=& \min [\nonumber\\
&&\min( \left(\rho q\right)^{\dagger}_{i,j,k},\left(\rho q\right)^{n}_{i,j,k} ),\nonumber\\
&&\min( \left(\rho q\right)^{\dagger}_{i-1,j,k},\left(\rho q\right)^{n}_{i-1,j,k} ),\nonumber\\
&&\min( \left(\rho q\right)^{\dagger}_{i+1,j,k},\left(\rho q\right)^{n}_{i+1,j,k} ),\nonumber\\
&&\min( \left(\rho q\right)^{\dagger}_{i,j-1,k},\left(\rho q\right)^{n}_{i,j-1,k} ),\nonumber\\
&&\min( \left(\rho q\right)^{\dagger}_{i,j+1,k},\left(\rho q\right)^{n}_{i,j+1,k} ),\nonumber\\
&&\min( \left(\rho q\right)^{\dagger}_{i,j,k-1},\left(\rho q\right)^{n}_{i,j,k-1} ),\nonumber\\
&&\min( \left(\rho q\right)^{\dagger}_{i,j,k+1},\left(\rho q\right)^{n}_{i,j,k+1} ) \nonumber\\
&&]
\end{eqnarray}
\item Several values for the flux limiter are calculated:
\begin{eqnarray}
P_{i,j,k}^{+} &=& 
-\min ( 0, A_{i+\frac{1}{2},j,k} ) + \max( 0, A_{i-\frac{1}{2},j,k} )\nonumber\\
&&
-\min ( 0, A_{i,j+\frac{1}{2},k} ) + \max( 0, A_{i,j-\frac{1}{2},k} )\nonumber\\
&&
-\min ( 0, A_{i,j,k+\frac{1}{2}} ) + \max( 0, A_{i,j,k-\frac{1}{2}} )\\
P_{i,j,k}^{-} &=& 
-\max ( 0, A_{i+\frac{1}{2},j,k} ) + \min( 0, A_{i-\frac{1}{2},j,k} )\nonumber\\
&&
-\max ( 0, A_{i,j+\frac{1}{2},k} ) + \min( 0, A_{i,j-\frac{1}{2},k} )\nonumber\\
&&
-\max ( 0, A_{i,j,k+\frac{1}{2}} ) + \min( 0, A_{i,j,k-\frac{1}{2}} )\\
\end{eqnarray}
\begin{eqnarray}
Q_{i,j,k}^{+} &=& 
\left[\left(\rho q\right)^{\max}_{i,j,k} - \left(\rho q\right)^{\dagger}_{i,j,k} \right]
\Delta x \Delta y \Delta z\\
Q_{i,j,k}^{-} &=& 
\left[\left(\rho q\right)^{\dagger}_{i,j,k}-\left(\rho q\right)^{\min}_{i,j,k} \right]
\Delta x \Delta y \Delta z
\end{eqnarray}
\begin{eqnarray}
R_{i,j,k}^{+} &=& 
\begin{cases}
        \min( 1, Q_{i,j,k}^{+}/P_{i,j,k}^{+} ) & {\rm if~} P_{i,j,k}^{+}>0\\
        0 & {\rm if~} P_{i,j,k}^{+}=0\\
\end{cases}
\\
R_{i,j,k}^{-} &=& 
\begin{cases}
        \min( 1, Q_{i,j,k}^{-}/P_{i,j,k}^{-} ) & {\rm if~} P_{i,j,k}^{-}>0\\
        0 & {\rm if~} P_{i,j,k}^{-}=0\\
\end{cases}
\end{eqnarray}
\item The flux limters at the cell wall are calculated:
\begin{eqnarray}
C_{i+\frac{1}{2},j,k} &=& 
\begin{cases}
        \min( R_{i+1,j,k}^{+}, R_{i,j,k}^{-} ) & {\rm if~} A_{i+\frac{1}{2},j,k}^{-}\geq 0\\
        \min( R_{i,j,k}^{+}, R_{i+1,j,k}^{-} ) & {\rm if~} A_{i+\frac{1}{2},j,k}^{-}<0\\
\end{cases}
\\
C_{i,j+\frac{1}{2},k} &=& 
\begin{cases}
        \min( R_{i,j+1,k}^{+}, R_{i,j,k}^{-} ) & {\rm if~} A_{i,j+\frac{1}{2},k}^{-}\geq 0\\
        \min( R_{i,j,k}^{+}, R_{i,j+1,k}^{-} ) & {\rm if~} A_{i,j+\frac{1}{2},k}^{-}<0\\
\end{cases}
\\
C_{i,j,k+\frac{1}{2}} &=& 
\begin{cases}
        \min( R_{i,j,k+1}^{+}, R_{i,j,k}^{-} ) & {\rm if~} A_{i,j,k+\frac{1}{2}}^{-}\geq 0\\
        \min( R_{i,j,k}^{+}, R_{i,j,k+1}^{-} ) & {\rm if~} A_{i,j,k+\frac{1}{2}}^{-}<0\\
\end{cases}
\end{eqnarray}
\end{enumerate}

